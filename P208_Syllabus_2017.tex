\documentclass[12pt]{article}

%%%%%%%%%%%%%%%%%%%%%%%%%%%%%
%%%  THE USUAL PACKAGES  %%%%
%%%%%%%%%%%%%%%%%%%%%%%%%%%%%

\usepackage{amsmath}
\usepackage{amssymb}
\usepackage{amsfonts}
\usepackage{graphicx}
\usepackage{xcolor}
\usepackage{nopageno}

%%%%%%%%%%%%%%%%%%%%%%%%%%%%%%%%%
%%%  UNUSUAL PACKAGES        %%%%
%%%  Uncomment as necessary. %%%%
%%%%%%%%%%%%%%%%%%%%%%%%%%%%%%%%%

%% MATH AND PHYSICS SYMBOLS
%% ------------------------
%\usepackage{slashed}       % \slashed{k}
%\usepackage{mathrsfs}      % Weinberg-esque letters
%\usepackage{youngtab}	    % Young Tableaux
%\usepackage{pifont}        % check marks
%\usepackage{bbm}           % \mathbbm{1} incomp. w/ XeLaTeX 
%\usepackage[normalem]{ulem} % for \sout


%% CONTENT FORMAT AND DESIGN (below for general formatting)
%% --------------------------------------------------------
\usepackage{lipsum}        % block of text (formatting test)
%\usepackage{color}         % \color{...}, colored text
%\usepackage{framed}        % boxed remarks
%\usepackage{subcaption}    % subfigures; subfig depreciated
%\usepackage{paralist}      % compactitem
%\usepackage{appendix}      % subappendices
%\usepackage{cite}          % group cites (conflict: collref)
%\usepackage{tocloft}       % Table of Contents	

%% TABLES IN LaTeX
%% ---------------
%\usepackage{booktabs}      % professional tables
%\usepackage{nicefrac}      % fractions in tables,
%\usepackage{multirow}      % multirow elements in a table
%\usepackage{arydshln} 	    % dashed lines in arrays

%% Other Packages and Notes
%% ------------------------
%\usepackage[font=small]{caption} % caption font is small





%%%%%%%%%%%%%%%%%%%%%%%%%%%%%%%%%%%%%%%%%%%%%%%
%%%  PAGE FORMATTING and (RE)NEW COMMANDS  %%%%
%%%%%%%%%%%%%%%%%%%%%%%%%%%%%%%%%%%%%%%%%%%%%%%

\usepackage[margin=2cm]{geometry}   % reasonable margins

\graphicspath{{figures/}}	        % set directory for figures

% for capitalized things
\newcommand{\acro}[1]{\textsc{\MakeLowercase{#1}}}    

\numberwithin{equation}{section}    % set equation numbering
\renewcommand{\tilde}{\widetilde}   % tilde over characters
\renewcommand{\vec}[1]{\mathbf{#1}} % vectors are boldface

\newcommand{\dbar}{d\mkern-6mu\mathchar'26}    % for d/2pi
\newcommand{\ket}[1]{\left|#1\right\rangle}    % <#1|
\newcommand{\bra}[1]{\left\langle#1\right|}    % |#1>
\newcommand{\Xmark}{\text{\sffamily X}}        % cross out

% Change list spacing (instead of package paralist)
% from: http://en.wikibooks.org/wiki/LaTeX/List_Structures#Line_spacing
\let\oldenumerate\enumerate
\renewcommand{\enumerate}{
  \oldenumerate
  \setlength{\itemsep}{1pt}
  \setlength{\parskip}{0pt}
  \setlength{\parsep}{0pt}
}

\let\olditemize\itemize
\renewcommand{\itemize}{
  \olditemize
  \setlength{\itemsep}{1pt}
  \setlength{\parskip}{0pt}
  \setlength{\parsep}{0pt}
}


% Commands for temporary comments
\newcommand{\comment}[2]{\textcolor{red}{[\textbf{#1} #2]}}
\newcommand{\flip}[1]{{\color{red} [\textbf{Flip}: {#1}]}}
\newcommand{\email}[1]{\texttt{\href{mailto:#1}{#1}}}

\newenvironment{institutions}[1][2em]{\begin{list}{}{\setlength\leftmargin{#1}\setlength\rightmargin{#1}}\item[]}{\end{list}}


\usepackage{fancyhdr}		% to put preprint number



%%%%%%%%%%%%%%%%%%%
%%%  HYPERREF  %%%%
%%%%%%%%%%%%%%%%%%%

%% This package has to be at the end; can lead to conflicts
\usepackage{microtype}
\usepackage[
	colorlinks=true,
	citecolor=black,
	linkcolor=black,
	urlcolor=green!50!black,
	hypertexnames=false]{hyperref}



%%%%%%%%%%%%%%%%%%%%%
%%%  TITLE DATA  %%%%
%%%%%%%%%%%%%%%%%%%%%

%%% PREPRINT NUMBER USING fancyhdr
%%% Don't forget to set \thispagestyle{firststyle}
%%% ----------------------------------------------
%\renewcommand{\headrulewidth}{0pt} % no separator
%\fancypagestyle{firststyle}{
%\rhead{\footnotesize \texttt{UCI-TR-2016-XX}}}



\begin{document}

%\thispagestyle{empty}
%\thispagestyle{firststyle} %% to include preprint

\begin{center}

    {\Large \textsc{Physics 208:} \textbf{General Relativity}}
    
\end{center}

\vskip .4cm

\noindent
\begin{tabular*}{\textwidth}{rlcrll}
	\textsc{Instructor:}& Flip Tanedo
	&
	\hspace{1.2cm}
	&
	\textsc{Room:} & Physics & 2104
%	\textsc{Term:}& %\multicolumn{2}{l}{Fall Quarter 2016}
%	F2016
	\\
	\textsc{Contact:}& \email{flip.tanedo@ucr.edu} 
	&
	\hfill
	&
	\textsc{Lecture:}& TR & 5:10pm -- 6:30pm
	\\
	\textsc{Office:}& Physics 3054
	&
	\hfill
	&
	\textsc{Office Hour:}&  & By appointment
\end{tabular*}





%\subsection*{Official Course Description}
%\begin{quote}
%	\textsc{Lecture, 3 hours; discussion, 1 hour.} Prerequisite(s): \textsc{phys}~205. Tensors, covariant derivative, the Riemann curvature tensor and Einstein’s equation. The Schwartzchild metric and applications to the solar system and black holes. Gravity waves and expanding universe.
%\end{quote}


% This course description is a piece of shit
% (1) inconsistent pluralization
% (2) misspelling `Schwarzschild'
% (3) gravity waves are not `gravitational waves'

\subsection*{Course Description}

%This course is a survey of the geometric theory of spacetime: review of Minkowski space and special relativity, covariant derivatives, tensors, geodesics, Schwarzchild geometry, black holes, and gravitational waves.

This course is a survey of the geometric theory of spacetime and its physical implications. Due to the restrictions of a 10~week quarter and in the interest of making this course broadly accessible, we take the `Santa Barbara' approach of developing differential geometry tools as needed. 
%
We will understand the physics of curved spacetime, including black holes and the recent discovery of gravitational waves.


\subsection*{Course Materials}

All course materials will be accessible through the \href{http://faculty.ucr.edu/~flipt/P208_2016.html}{course webpage}\footnote{\url{http://faculty.ucr.edu/~flipt/P208_2016.html}}. Announcements and grades will be managed, begrudgingly, through \texttt{iLearn}.


\subsection*{Evaluation}

Weekly homework assignments that may include small projects and \emph{will include independent reading}. If you are taking this class, you already understand that real learning only occurs when one tackles actual problems; please do these assignments. No exams. I expect you to work together and to abide by the \href{http://conduct.ucr.edu/policies/academicintegrity.html}{UCR academic integrity policies}.

\subsection*{Textbook}

The course textbook is \emph{Gravity: An Introduction to Einstein's General Relativity} (\textsc{isbn} \texttt{0-8053-8662-9}) by James Hartle. This book is straightforward, physically grounded, and is mathematically accessible. 
%
There are two other books which I strongly recommend depending on what you intend to get out of this course:
\begin{enumerate}
	\item \emph{Spacetime and Geometry: An Introduction to General Relativity} by Sean Carroll; based on \texttt{gr-qc/9712019}. Excellent for those who want to see more explicitly how differential geometry underpins this subject. Be sure to read the appendices. If we had two quarters for this course, we would use this textbook.
	\item \emph{Einstein Gravity in a Nutshell}, Anthony Zee. Like Zee's other `in a nutshell' textbooks, this is a delight to read---even more so after you have some background in the subject. The text is conversational and excellent for meandering self-study, but the unconventional ordering may be difficult for our short course.
% Weinberg
% Dirac
% Ryder?
\end{enumerate}

\vspace{.5em}
\noindent Refer to the course webpage for additional references. Feel free to also consult other references. Let me know if you find anything particularly useful.

\section*{Topics}

This is a preliminary partitioning of topics; I reserve the right to update this as necessary. Leftover weeks are for make-up lectures.

\begin{enumerate}
	\item \textbf{Flat Spacetime} [3 lecs]. Hartle, Chapters 2 -- 5. Spacetime as a metric space. Review of special relativity. Principle of relativity. Observers. Tensors.

	\item \textbf{Gravity as Geometry} [4 lecs]. Hartle, Chapters 6 -- 8. Curved space, the equivalence principle. Geodesic equation, locally inertial frames. Differential geometry as needed.
	
	\item \textbf{Schwarzschild Solution} [4 lecs]. Hartle, Chapters 9, 11, 12. Black holes.

	\item \textbf{Some Formalism} [4 lecs]. Hartle, Chapters 20 -- 22. Need-to-know differential geometry, curvature, the Einstein equation.


	\item \textbf{Gravitational Waves} [4 lecs]. Hartle, Chapters 16 and 23. Gravitational radiation and the 2015 LIGO discovery.
\end{enumerate}


\end{document}